\documentclass{article}
\usepackage[utf8]{inputenc}
\usepackage[russian]{babel}
\usepackage{cmap}

\begin{document}
В выпуклой оболочке столько же вершин, сколько и сторон в случае, если она невырожденная.
Поэтому задача сводится к тому, чтобы для каждой пары несовпадающих точек $A$ и $B$ определить, сколько подмножеств содержат $AB$ как сторону выпуклой оболочки, а также обработать вырожденные случаи.

Объединим совпадающие точки из входных данных в одну с весом, равным количеству совпадающих.
Далее везде будем считать, что точка~--- это не только координаты, но еще и вес.

Зафиксируем $A$. Все остальные точки упорядочим по полярному углу относительно $A$ (можно, например, считать, что точки верхней полуплоскости будут идти раньше точек нижней).
Для каждой точки нам будет важно только расстояние (квадрат расстояния) до A и ее вес, координаты можно забыть.

Возьмем самую первую точку $B_0$.
Обозначим $S_1$ и $S_2$~--- множества точек в верхней и нижней полуплоскостях от прямой $AB_0$.
Найдем веса $S_1$ и $S_2$, мы будем их поддерживать во время работы.

Будем обрабатывать вектора в заданном порядке, на каждом из них точки в порядке удаления от $A$.
Пусть текущая точка $B$, между $A$ и $B$ (не включая сами $A$ и $B$) множество точек $W$, его вес известен.
Также нам известны веса $S_1$ и $S_2$.

$AB$ лежит в невырожденной выпуклой оболочке в двух случаях:

1) Если мы берем произвольный непустой набор точек из $S_1$, произвольный из $W$ и $0$ точек из $S_2$.
$(2^{S_1} - 1) \cdot 2^W$ вариантов.

2) Аналогично, если берем произвольный непустой набор точек из $S_2$, произвольный из $W$ и $0$ точек из $S_1$.
$(2^{S_2} - 1) \cdot 2^W$ вариантов.

$AB$ лежит в вырожденной в отрезок выпуклой оболочке, когда мы не берем точек из $S_1$ и $S_2$, а только из $W$.
Соответственно, к ответу добавляется $2 \cdot 2^W$, т.к. в этой случае имеем $2$ вершины.

Перейдем к следующей точке, обновив веса $S_1$ и $S_2$, для чего нам нужно знать суммарный вес точек на текущем векторе и на обратном ему. Если мы остались на том же векторе, то к $W$ нужно прибавить вес $B$.

Чтобы не учитывать дважды отрезки $AB$ и $BA$, можно обходить только B из верхней полуплоскости.



\end{document}
